\documentclass[a4paper,12pt,dvipsnames]{article}
\usepackage[utf8x]{inputenc}

\usepackage[frenchb]{babel}
\usepackage[T1]{fontenc}

\usepackage{graphicx}
\usepackage{amssymb}
\usepackage{amsmath}
\usepackage{hyperref}% hyperliens
\usepackage{tikz}
\usetikzlibrary{babel,positioning,calc}
\usepackage[]{circuitikz}
\usepackage{textcomp}
\usepackage{gensymb} % \ohm, celsius
\usepackage{todo}
\usepackage{pgfplots}

\usepackage{mathastext} % math as standard text : units are respecting typography conventions.

\usepackage{minted}
\usepackage{xcolor}
\setminted{%
frame=single,
fontsize=\scriptsize,
breaklines,
}

\usepackage{fullpage}


%from SO: kinky cross for wires
\tikzset{
  declare function={% in case of CVS which switches the arguments of atan2
    atan3(\a,\b)=ifthenelse(atan2(0,1)==90, atan2(\a,\b), atan2(\b,\a));},
  kinky cross radius/.initial=+.125cm,
  @kinky cross/.initial=+, kinky crosses/.is choice,
  kinky crosses/left/.style={@kinky cross=-},kinky crosses/right/.style={@kinky cross=+},
  kinky cross/.style args={(#1)--(#2)}{
    to path={
      let \p{@kc@}=($(\tikztotarget)-(\tikztostart)$),
          \n{@kc@}={atan3(\p{@kc@})+180} in
      -- ($(intersection of \tikztostart--{\tikztotarget} and #1--#2)!%
             \pgfkeysvalueof{/tikz/kinky cross radius}!(\tikztostart)$)
      arc [ radius     =\pgfkeysvalueof{/tikz/kinky cross radius},
            start angle=\n{@kc@},
            delta angle=\pgfkeysvalueof{/tikz/@kinky cross}180 ]
      -- (\tikztotarget)}}}



\title{\vspace*{-5cm}TikZ tricks \\ Stop wasting your time on tex.stackexchange.com}

\begin{document}

\maketitle

\section{Basic circuits}

\subsection{Voltage source and lamp}
\begin{center}
\begin{circuitikz}\draw
	(0,0) to [sinusoidal voltage source, l=$E$] (0,2)--(2,2) to [lamp, l=$R$] (2,0) --(0,0);
\end{circuitikz}
\end{center}

\begin{minted}{latex}
		\begin{circuitikz}\draw
			(0,0) to [sinusoidal voltage source, l=$E$] (0,2)--(2,2) to [lamp, l=$R$] (2,0) --(0,0);
		\end{circuitikz}
\end{minted}











\section{Filters}

\subsection{RLC - Out on RL}
\begin{center}
\begin{circuitikz}[scale=0.8]\draw
	(0,0) to [open, v^>=$v_{in}$,o-o] (0,4) to [C,l=$22nF$] (2,4) to [L,l=$12mH$]  (2,2) to [R,l=$82\Omega$](2,0) to [short](0,0)
	(2,4) -- (4,4) to [open, v^<=$v_{out}$,o-o] (4,0) -- (2,0)
	;
\end{circuitikz}
\end{center}

\begin{minted}{latex}
		\begin{circuitikz}[scale=0.8]\draw
			(0,0) to [open, v^>=$v_{in}$,o-o] (0,4) to [C,l=$22nF$] (2,4) to [L,l=$12mH$]  (2,2) to [R,l=$82\Omega$](2,0) to [short](0,0)
			(2,4) -- (4,4) to [open, v^<=$v_{out}$,o-o] (4,0) -- (2,0);
		\end{circuitikz}
\end{minted}


\subsection{RC high-pass}
\begin{center}
\begin{circuitikz} \draw
(0,0)   node[ground]{}
		to[american voltage source, v=$V_{in}$, invert] (0,3)
		to[C, l=$C$] (3,3)
		(3,0) to[R, l=$R$, v=$V_{out}$] (3,3)
		(3,0)--(0,0)
;
\end{circuitikz}
\end{center}

\begin{minted}{latex}
		\begin{circuitikz} \draw
			(0,0)   node[ground]{}
			to[american voltage source, v=$V_{in}$, invert] (0,3)
			to[C, l=$C$] (3,3)
			(3,0) to[R, l=$R$, v=$V_{out}$] (3,3)
			(3,0)--(0,0);
		\end{circuitikz}
\end{minted}


\subsection{RC high-pass with generator}
\begin{center}
\begin{circuitikz} \draw
	(0,0)   node[ground]{}
	to[sinusoidal voltage source, v=$V{in}$] 	(0,3)
	to[R, l=$R_G$]									(2,3)
	to[C, l=$10nF$]   						    (5,3)
	(5,0) to[R, l=$10k\Omega$, v=$V{out}$] (5,3)
	(5,0)--(0,0)
	(0,4.5) node[] {Générateur};
\draw[dotted](-2,-1)--(-2,4)--(2,4)--(2,-1)--(-2,-1);
\end{circuitikz}
\end{center}

\begin{minted}{latex}
		\begin{circuitikz} \draw
			(0,0)   node[ground]{}
			to[sinusoidal voltage source, v=$V{in}$] (0,3)
			to[R, l=$R_G$] (2,3)
			to[C, l=$10nF$] (5,3)
			(5,0) to[R, l=$10k\Omega$, v=$V{out}$] (5,3)
			(5,0)--(0,0)
			(0,4.5) node[] {Générateur};
		\draw[dotted](-2,-1)--(-2,4)--(2,4)--(2,-1)--(-2,-1);
		\end{circuitikz}
\end{minted}


\subsection{RLC - Out on C}
\begin{center}
\begin{circuitikz} \draw
	(0,0)   node[ground]{}
	to[american voltage source, v=$V_{in}$, invert] (0,3)
	to[R, l=$R$] (3,3)
	to[L, l=$L$] (6,3)
	(6,0) to[C, l=$C$, v=$V_{out}$] (6,3)
	(6,0)--(0,0);
\end{circuitikz}
\end{center}

\begin{minted}{latex}
		\begin{circuitikz} \draw
			(0,0)   node[ground]{}
			to[american voltage source, v=$V_{in}$, invert] (0,3)
			to[R, l=$R$] (3,3)
			to[L, l=$L$] (6,3)
			(6,0) to[C, l=$C$, v=$V_{out}$] (6,3)
			(6,0)--(0,0);
		\end{circuitikz}
\end{minted}


\subsection{RLC with generator - Out on C}
\begin{center}
\begin{circuitikz} \draw
	(0,0)   node[ground]{}
	to[sinusoidal voltage source, v=$V{in}$] (0,3)
	to[R, l=$R_G$] (2,3)
	to[R, l=$R$] (4,3)
	to[L, l=$8.2mH$] (6,3)
	(6,0) to[C, l=$33nF$, v=$V_{out}$] (6,3)
	(6,0)--(0,0)
	(0,4.5) node[] {Générateur};
	\draw[dotted](-2,-1)--(-2,4)--(2,4)--(2,-1)--(-2,-1);
\end{circuitikz}
\end{center}

\begin{minted}{latex}
		\begin{circuitikz} \draw
			(0,0)   node[ground]{}
			to[sinusoidal voltage source, v=$V{in}$] (0,3)
			to[R, l=$R_G$] (2,3)
			to[R, l=$R$] (4,3)
			to[L, l=$8.2mH$] (6,3)
			(6,0) to[C, l=$33nF$, v=$V_{out}$] (6,3)
			(6,0)--(0,0)
			(0,4.5) node[] {Générateur};
			\draw[dotted](-2,-1)--(-2,4)--(2,4)--(2,-1)--(-2,-1);
		\end{circuitikz}
\end{minted}
















\section{Transistors}

\subsection{Alone}
\begin{circuitikz} \draw
	(2.25, 1) node[nfet] (mos) {}		
	(mos.D) -- (2.25, 2) to  [short, -o](3.25, 2)  node[anchor=west] {} %D		
	(mos.S) -- (2.25, 0) to [short, -o](3.25, 0)  node[anchor=west] { } %S		
	(mos.B) -- (mos.S)
	(2.25,0) to [short, -o](0,0)  node[anchor=east] {} %S		
	(0,2)  node[anchor=east]{}[short, o-] to  (1,2) %G
	(1,2) -- (1,1) -- (mos.G)
	;
\end{circuitikz}\hspace*{1cm}
\begin{circuitikz}\draw
	(0,0) node[anchor=east] {} %g
	to [short, o-] (1,0) 
	to [open, v<={~}] (1,-2)
	to [short, -o] (4,-2)
	to [short, -o] (0,-2) node[anchor=east] {} %s
	(3,0) to [cI, i={~}] (3,-2)
	(3,-2) to [short, -o] (4,-2) node[anchor=west] {} %s
	(3,0) to [short, -o] (4,0)
		to node[anchor=west] {} (4,0) %d
;\end{circuitikz}

\begin{minted}{latex}
		\begin{circuitikz} \draw
			(2.25, 1) node[nfet] (mos) {}		
			(mos.D) -- (2.25, 2) to  [short, -o](3.25, 2)  node[anchor=west] {} %D		
			(mos.S) -- (2.25, 0) to [short, -o](3.25, 0)  node[anchor=west] { } %S		
			(mos.B) -- (mos.S)
			(2.25,0) to [short, -o](0,0)  node[anchor=east] {} %S		
			(0,2)  node[anchor=east]{}[short, o-] to  (1,2) %G
			(1,2) -- (1,1) -- (mos.G)
			;
		\end{circuitikz}\hspace*{1cm}
		\begin{circuitikz}\draw
			(0,0) node[anchor=east] {} %g
			to [short, o-] (1,0) 
			to [open, v<={~}] (1,-2)
			to [short, -o] (4,-2)
			to [short, -o] (0,-2) node[anchor=east] {} %s
			(3,0) to [cI, i={~}] (3,-2)
			(3,-2) to [short, -o] (4,-2) node[anchor=west] {} %s
			(3,0) to [short, -o] (4,0)
				to node[anchor=west] {} (4,0) %d
		;\end{circuitikz}
\end{minted}

\subsection{Alone with voltage and current} 
\begin{center}
	\begin{circuitikz} \draw
	(2.25, 1) node[nfet] (mos) {}	
	(mos.D) -- (2.25, 2) to  [short, -o, i<=$I_D$](3.25, 2)  node[anchor=west] {D} %D		
	(mos.S) -- (2.25, 0) to [short, -o](3.25, 0)  node[anchor=west] {S} %S		
	(mos.B) -- (mos.S)
	(2.25,0) to [short, -o](0,0)  node[anchor=east] {S} %S		
	(0,2)  node[anchor=east]{G}[short, o-] to  (1,2) %G
	(1,2) -- (1,1) -- (mos.G)
	(0,0) [open,v^>=$V_{GS}$] to (0,2)
	(3.25,0) [open,v>=$V_{DS}$] to (3.25,2)
	;\end{circuitikz}\hspace*{1cm}
	\begin{circuitikz}\draw
	(0,0) node[anchor=east] {g} %g
	to [short, o-] (1,0) 
	to [open, v<=$v_{gs}$] (1,-2)
	to [short, -o] (4,-2)
	to [short, -o] (0,-2) node[anchor=east] {s} %s
	(3,0) to [cI, i_=\rotatebox{90}{$g_m\cdot v_{gs}$}] (3,-2)
	(3,-2) to [short, -o] (4,-2) node[anchor=west] {s} %s
	(3,0) to [short, -o] (4,0)
	to node[anchor=west] {d} (4,0) %d
	(4.0,-2) [open,v>=$v_{ds}$] to (4.0,0)
;\end{circuitikz}	
\end{center}

\begin{center}
\begin{circuitikz}\draw
	(0,0) node[anchor=east] {g} 
	to [short, o-] (1,0) 
	to [open, v<=$v_{gs}$] (1,-2)
	to [short, -o] (0,-2)
	to  (0,-2) node[anchor=east] {s}
	(3,0) to [cI=$ g \cdot v_{gs}$] (3,-2)
	(3,-2) to [short, -o] (4,-2) node[anchor=west] {s}
	(3,0) to [short, -o] (4,0)
	to node[anchor=west] {d} (4,0)
	(1,-2) -- (3,-2)
;\end{circuitikz}
\end{center}

\begin{minted}{latex}
		\begin{circuitikz} \draw
			(2.25, 1) node[nfet] (mos) {}	
			(mos.D) -- (2.25, 2) to  [short, -o, i<=$I_D$](3.25, 2)  node[anchor=west] {D} %D		
			(mos.S) -- (2.25, 0) to [short, -o](3.25, 0)  node[anchor=west] {S} %S		
			(mos.B) -- (mos.S)
			(2.25,0) to [short, -o](0,0)  node[anchor=east] {S} %S		
			(0,2)  node[anchor=east]{G}[short, o-] to  (1,2) %G
			(1,2) -- (1,1) -- (mos.G)
			(0,0) [open,v^>=$V_{GS}$] to (0,2)
			(3.25,0) [open,v>=$V_{DS}$] to (3.25,2)
		;\end{circuitikz}\hspace*{1cm}
		\begin{circuitikz}\draw
			(0,0) node[anchor=east] {g} %g
			to [short, o-] (1,0) 
			to [open, v<=$v_{gs}$] (1,-2)
			to [short, -o] (4,-2)
			to [short, -o] (0,-2) node[anchor=east] {s} %s
			(3,0) to [cI, i_=\rotatebox{90}{$g_m\cdot v_{gs}$}] (3,-2)
			(3,-2) to [short, -o] (4,-2) node[anchor=west] {s} %s
			(3,0) to [short, -o] (4,0)
			to node[anchor=west] {d} (4,0) %d
			(4.0,-2) [open,v>=$v_{ds}$] to (4.0,0)
		;\end{circuitikz}

		\begin{circuitikz}\draw
			(0,0) node[anchor=east] {g} 
			to [short, o-] (1,0) 
			to [open, v<=$v_{gs}$] (1,-2)
			to [short, -o] (0,-2)
			to  (0,-2) node[anchor=east] {s}
			(3,0) to [cI=$ g \cdot v_{gs}$] (3,-2)
			(3,-2) to [short, -o] (4,-2) node[anchor=west] {s}
			(3,0) to [short, -o] (4,0)
			to node[anchor=west] {d} (4,0)
			(1,-2) -- (3,-2)
		;\end{circuitikz}
\end{minted}


\subsection{Full common source}
\begin{center}
	\begin{circuitikz}[scale=1]\draw
	(0,1) to [short,o-] (9,1)
	(4,6) to [short] (9,6)
	(0,3) node[anchor=east] {In} to [short,o-] (1,3)
	(0,3) node[anchor=south]{} to [open, v_<=$V_{in}$]  (0,1) 
	(1,3) to [C=$C_{in}$ ](1.5,3) 
	(1.5,3) to [short,-*] (2,3) node[anchor=south west]{}

	(2,6) node[anchor=south ] (alim) {$+V_{DC}$}
	(1.6,6) -- (2.4,6) %bar under the label
	(2,3) to [R, l_=$R_{B1}$](2,6)
	(2,3) to [R=$R_{B2}$](2,1)
	(4,3) node[nfet] (mos) {}
	(mos.G) to [short] (2,3)
	(mos.D) to (4,4) to [R, l_=$R_D$] (4, 6)		
	(mos.D) to [short,-*](4,3.5)  to [short] (4.25,3.5)
	(mos.S) to [short] (4,1)% to [short, -o](2,0)  node[anchor=west] {S}
	(mos.S) -- (mos.B) %source to bulk connection		

	(4.25,3.5) node[anchor=south]{} to [C, l^=$C{out}$] (6,3.5) to  [short](6,3.5)node[anchor=south]{} to [short,-o](6.5,3.5)node [anchor=south] {Out}	
	(6,3.5) to [generic, l_=$R_{ch}$] (6,1)
	(6.5,3.5) to [open,v^<=$V_{out}$] (6.5,1)
	(9,6) to [battery, l_=$E$](9,1)
	(4,1) node[circ]{}
	(4,1) node[ground]{}
	;\end{circuitikz}
\end{center}


\begin{minted}{latex}
		\begin{circuitikz}[scale=1]\draw
		(0,1) to [short,o-] (9,1)
		(4,6) to [short] (9,6)
		(0,3) node[anchor=east] {In} to [short,o-] (1,3)
		(0,3) node[anchor=south]{} to [open, v_<=$V_{in}$]  (0,1) 
		(1,3) to [C=$C_{in}$ ](1.5,3) 
		(1.5,3) to [short,-*] (2,3) node[anchor=south west]{}

		(2,6) node[anchor=south ] (alim) {$+V_{DC}$}
		(1.6,6) -- (2.4,6) %bar under the label
		(2,3) to [R, l_=$R_{B1}$](2,6)
		(2,3) to [R=$R_{B2}$](2,1)
		(4,3) node[nfet] (mos) {}
		(mos.G) to [short] (2,3)
		(mos.D) to (4,4) to [R, l_=$R_D$] (4, 6)		
		(mos.D) to [short,-*](4,3.5)  to [short] (4.25,3.5)
		(mos.S) to [short] (4,1)% to [short, -o](2,0)  node[anchor=west] {S}
		(mos.S) -- (mos.B) %source to bulk connection		

		(4.25,3.5) node[anchor=south]{} to [C, l^=$C{out}$] (6,3.5) to  [short](6,3.5)node[anchor=south]{} to [short,-o](6.5,3.5)node [anchor=south] {Out}	
		(6,3.5) to [generic, l_=$R_{ch}$] (6,1)
		(6.5,3.5) to [open,v^<=$V_{out}$] (6.5,1)
		(9,6) to [battery, l_=$E$](9,1)
		(4,1) node[circ]{}
		(4,1) node[ground]{}
		;\end{circuitikz}
\end{minted}


\subsection{Common source - Direct polarisation}
\begin{center}
\begin{circuitikz}[scale=1]\draw
	(0,1) to [short,o-] (9,1)
	(4,6) to [short] (9,6)
	(0,3) node[anchor=east] {In} to [short,o-] (1,3)
	(0,3) to [open, v_<=$V_{in}$]  (0,1)
	(1,3) to [C=$C_{in}$ ](1.5,3)
	(1.5,3) to [short,-*] (2,3)
	(2,6) node[anchor=south ] (alim) {$+V_{DC}$}
	(1.6,6) -- (2.4,6) %bar under the label
	(2,3) to [R, l_=$R_{B1}$](2,6)
	(4,3) node[nfet] (mos) {}
	(mos.G) to [short] (2,3)
	(mos.D) to (4,4) to [R, l_=$R_D$] (4, 6)	
	(mos.D) to [short,-*](4,3.5)  to [short] (4.25,3.5)
	(mos.S) to [short] (4,1)% to [short, -o](2,0)  node[anchor=west] {S}
	(mos.S) -- (mos.B) %source to bulk connection		

	(4.25,3.5) to [C, l^=$C{out}$] (6,3.5) to  [short](6,3.5) to [short,-o](6.5,3.5)node [anchor=south] {Out}	
	(6,3.5) to [generic, l_=$Z_{ch}$] (6,1)
	(6.5,3.5) to [open,v^<=$V_{out}$] (6.5,1)
	(9,6) to [battery, l=$E$](9,1)
;\end{circuitikz}
\end{center}


\begin{minted}{latex}
	\begin{circuitikz}[scale=1]\draw
		(0,1) to [short,o-] (9,1)
		(4,6) to [short] (9,6)
		(0,3) node[anchor=east] {In} to [short,o-] (1,3)
		(0,3) to [open, v_<=$V_{in}$]  (0,1)
		(1,3) to [C=$C_{in}$ ](1.5,3)
		(1.5,3) to [short,-*] (2,3)
		(2,6) node[anchor=south ] (alim) {$+V_{DC}$}
		(1.6,6) -- (2.4,6) %bar under the label
		(2,3) to [R, l_=$R_{B1}$](2,6)
		(4,3) node[nfet] (mos) {}
		(mos.G) to [short] (2,3)
		(mos.D) to (4,4) to [R, l_=$R_D$] (4, 6)	
		(mos.D) to [short,-*](4,3.5)  to [short] (4.25,3.5)
		(mos.S) to [short] (4,1)% to [short, -o](2,0)  node[anchor=west] {S}
		(mos.S) -- (mos.B) %source to bulk connection		

		(4.25,3.5) to [C, l^=$C{out}$] (6,3.5) to  [short](6,3.5) to [short,-o](6.5,3.5)node [anchor=south] {Out}	
		(6,3.5) to [generic, l_=$Z_{ch}$] (6,1)
		(6.5,3.5) to [open,v^<=$V_{out}$] (6.5,1)
		(9,6) to [battery, l=$E$](9,1)
	;\end{circuitikz}
\end{minted}


\subsection{Common source - small signal}
\begin{center}
\begin{circuitikz}[scale=0.8]\draw
	(1,0) to [short,o-o] (11,0)
	(1,3) node[anchor=east] {In} to [short,o-] (1,3)
	(1,3) to [open, v_<=$V_{in}$]  (1,0)
	(1,3) to [short] (3,3)
	(2,3) to [R, l_=$R_{b1}$](2,0)
	(3,3) to [R=$R_{b2}$](3,0)
	(3,3) to [short,-o](4,3) node [anchor=west] {} 
	(4,3) to [open, v^<=$v_{g}$](4,0)
	(6,3) to [cI=\rotatebox{90}{$g_m \cdot v_{g}$}] (6,0)
	(8.5,0) to [R,l_=$R_D$] (8.5,3)
	(10,3) to [generic, l=$R_{ch}$] (10,0)
	(6,3) to [short,-o] (11,3) node [anchor=west] {Out}
	(11,3) to [open, v^<=$V_{out}$](11,0)
;\end{circuitikz}
\end{center}

\begin{minted}{latex}
		\begin{circuitikz}[scale=0.8]\draw
			(1,0) to [short,o-o] (11,0)
			(1,3) node[anchor=east] {In} to [short,o-] (1,3)
			(1,3) to [open, v_<=$V_{in}$]  (1,0)
			(1,3) to [short] (3,3)
			(2,3) to [R, l_=$R_{b1}$](2,0)
			(3,3) to [R=$R_{b2}$](3,0)
			(3,3) to [short,-o](4,3) node [anchor=west] {} 
			(4,3) to [open, v^<=$v_{g}$](4,0)
			(6,3) to [cI=\rotatebox{90}{$g_m \cdot v_{g}$}] (6,0)
			(8.5,0) to [R,l_=$R_D$] (8.5,3)
			(10,3) to [generic, l=$R_{ch}$] (10,0)
			(6,3) to [short,-o] (11,3) node [anchor=west] {Out}
			(11,3) to [open, v^<=$V_{out}$](11,0)
		;\end{circuitikz}
\end{minted}



\subsection{Common source simple}
\begin{center}
\ctikzset{tripoles/mos style/arrows}
\begin{circuitikz}[scale=0.8]\draw
	(0,0) to [V=$e(t)$] (0,2)
	(0,2) to [short] (1,2)
	(0,0) to (1,0)
	(1,2) to [open, v^<=$V_{GS}$](1,0)
	(1,0) to [short, o-] (2,0)
	(3,2) node[nigfete ] (mos) {}
	(3,0) to [short] (mos.S)
	(1,2) to [short] (mos.G)
	(2,0) to (3,0)
	(mos.D) to [short, i<=$I_D$](3,3)
	(3,3) to [R,l=$R_D$ ] (3,5)
	(3,5) to (4,5)
	(2,0) -- (5,0)
	(5,5) -- (3,5)
	(5,5) to [battery, l=$12V$] (5,0)
	(3,0) node[ground] {}
;\end{circuitikz}
\end{center}

\begin{minted}{latex}
		\ctikzset{tripoles/mos style/arrows}
		\begin{circuitikz}[scale=0.8]\draw
			(0,0) to [V=$e(t)$] (0,2)
			(0,2) to [short] (1,2)
			(0,0) to (1,0)
			(1,2) to [open, v^<=$V_{GS}$](1,0)
			(1,0) to [short, o-] (2,0)
			(3,2) node[nigfete ] (mos) {}
			(3,0) to [short] (mos.S)
			(1,2) to [short] (mos.G)
			(2,0) to (3,0)
			(mos.D) to [short, i<=$I_D$](3,3)
			(3,3) to [R,l=$R_D$ ] (3,5)
			(3,5) to (4,5)
			(2,0) -- (5,0)
			(5,5) -- (3,5)
			(5,5) to [battery, l=$12V$] (5,0)
			(3,0) node[ground] {}
		;\end{circuitikz}
\end{minted}


\subsection{Common source simple with $v_{out}$}

\begin{center}
\begin{circuitikz}[scale=0.8] \draw
	(0,0) to [sV=$e(t)$] (0,2)
	(0,2) to [short] (1,2)
	(0,0) to (1,0)
	(1,2) to [open, v^<=$v_{in}$](1,0)
	(1,0) to [short, o-] (2,0)
	(3,2) node[nigfete ] (mos) {}
	(mos.S) to [short] (3,0)
	(1,2) to [short] (mos.G)
	(2,0) to (3,0)
	(mos.D) to [short](3,3) %, i<=$I_D$
	(3,3) to [R, l=$330\ohm$] (3,5)
	(3,3) to [short, -o](4,3)
	(4,3) node[anchor=west] {$v_{out}$}
	(3,5) node[rground, yscale=-1] (alim) {}
	(3,5.7) node {+12V}
	(3,0) node[ground] {}
;\end{circuitikz}
\end{center}

\begin{minted}{latex}
\begin{circuitikz}[scale=0.8] \draw
	(0,0) to [sV=$e(t)$] (0,2)
	(0,2) to [short] (1,2)
	(0,0) to (1,0)
	(1,2) to [open, v^<=$v_{in}$](1,0)
	(1,0) to [short, o-] (2,0)
	(3,2) node[nigfete ] (mos) {}
	(mos.S) to [short] (3,0)
	(1,2) to [short] (mos.G)
	(2,0) to (3,0)
	(mos.D) to [short](3,3) %, i<=$I_D$
	(3,3) to [R, l=$330\ohm$] (3,5)
	(3,3) to [short, -o](4,3)
	(4,3) node[anchor=west] {$v_{out}$}
	(3,5) node[rground, yscale=-1] (alim) {}
	(3,5.7) node {+12V}
	(3,0) node[ground] {}
;\end{circuitikz}
\end{minted}




















\section{Operational amplifiers}

\subsection{Inverter with voltage and buffered offset}

\begin{circuitikz} [scale=1.2]\draw
	(0,0) node[op amp] (opamp) {}
	(opamp.down) ++ (0,-0.5) node[ground]{} -- (opamp.down)
	(opamp.up) ++ (0,.5) node[above] {12V} -- (opamp.up)
	(opamp.-) -| (-1.5,2) to [R, l=$R2$] (1.5,2) |-  (opamp.out)
	(opamp.+) -| (-1.5,-0.4) to [european voltage source, l_=$V_{C}$,-*] (-1.5,-2) node[ground] {} 
	(-4,-2) node[ground] {}  to [sV,*-*] (-4,0.4) |- ++(0.5,0) to [C,l=$C1$] ++(0.25,0) to [R,l=$R1$] (opamp.-)
	(-4,-2) node[anchor=west] {$0V$}
	(-1.5,-2) node[anchor=west] {$0V$}
	(-2.9,0.4) node[circ]{}
	(-2.9,0.4) node[anchor=south]{\rotatebox{90}{$6.3V+v_{in}$}}
	(-1.5,0.4) node[circ]{}
	(-1.5,0.4) node[anchor=south west]{\rotatebox{42}{$6.3V$}}
	(-1.5,-0.4) node[circ]{}
	(-1.5,-0.4) node[anchor=east]{$6.3V$}
	(1.5,0) node[circ]{}
	(1.5,0) node[anchor=south west]{$6.3V-10v_{in}$}
	(opamp.out) to (2.5,0)
	(2.5,-2) node[ground] {} to [open, v>=$V_{out}$] (2.5,0)
	(-4.5,-2) to [open, v^>=$v_{in}$] (-4.5,0.5)
	(-4,0.4) node[anchor=east] {$v_{in}$}
;\end{circuitikz}


\begin{minted}{latex}
		\begin{circuitikz} [scale=1.2]\draw
			(0,0) node[op amp] (opamp) {}
			(opamp.down) ++ (0,-0.5) node[ground]{} -- (opamp.down)
			(opamp.up) ++ (0,.5) node[above] {12V} -- (opamp.up)
			(opamp.-) -| (-1.5,2) to [R, l=$R2$] (1.5,2) |-  (opamp.out)
			(opamp.+) -| (-1.5,-0.4) to [european voltage source, l_=$V_{C}$,-*] (-1.5,-2) node[ground] {} 
			(-4,-2) node[ground] {}  to [sV,*-*] (-4,0.4) |- ++(0.5,0) to [C,l=$C1$] ++(0.25,0) to [R,l=$R1$] (opamp.-)
			(-4,-2) node[anchor=west] {$0V$}
			(-1.5,-2) node[anchor=west] {$0V$}
			(-2.9,0.4) node[circ]{}
			(-2.9,0.4) node[anchor=south]{\rotatebox{90}{$6.3V+v_{in}$}}
			(-1.5,0.4) node[circ]{}
			(-1.5,0.4) node[anchor=south west]{\rotatebox{42}{$6.3V$}}
			(-1.5,-0.4) node[circ]{}
			(-1.5,-0.4) node[anchor=east]{$6.3V$}
			(1.5,0) node[circ]{}
			(1.5,0) node[anchor=south west]{$6.3V-10v_{in}$}
			(opamp.out) to (2.5,0)
			(2.5,-2) node[ground] {} to [open, v>=$V_{out}$] (2.5,0)
			(-4.5,-2) to [open, v^>=$v_{in}$] (-4.5,0.5)
			(-4,0.4) node[anchor=east] {$v_{in}$}
		;\end{circuitikz}
\end{minted}





\begin{center}
\begin{circuitikz}\draw
	(0,4.5) to [photodiode,v_<=$V_D$, ] (0,0) node [ground] {}
	(4,4) node[op amp, yscale=-1] (opamp) {}
	(opamp.down) ++ (0,+0.5) node[above] {12V} -- (opamp.down)
	(opamp.up) ++ (0,-0.5) node[below] {-12V} -- (opamp.up)
	(opamp.-) -| ++(-0.5,-1.5) to [R, l_=$R_2$] ++(2.75,0) -|  (opamp.out)
	(opamp.-) -| ++(-0.5,-1.5) to [R, l=$R_1$] (2.25,0) node[ground] {}
	(opamp.+) to [short](0,4.5)
	(opamp.out) to [short] ++(1.5,0) node (A) {}
	to [open, v^<=$V_{amp}$, o-o] ++(0,-4) node [ground]{}
	;
\end{circuitikz}
\end{center}


\begin{minted}{latex}
\begin{circuitikz}\draw
	(0,4.5) to [photodiode,v_<=$V_D$, ] (0,0) node [ground] {}
	(4,4) node[op amp, yscale=-1] (opamp) {}
	(opamp.down) ++ (0,+0.5) node[above] {12V} -- (opamp.down)
	(opamp.up) ++ (0,-0.5) node[below] {-12V} -- (opamp.up)
	(opamp.-) -| ++(-0.5,-1.5) to [R, l_=$R_2$] ++(2.75,0) -|  (opamp.out)
	(opamp.-) -| ++(-0.5,-1.5) to [R, l=$R_1$] (2.25,0) node[ground] {}
	(opamp.+) to [short](0,4.5)
	(opamp.out) to [short] ++(1.5,0) node (A) {}
	to [open, v^<=$V_{amp}$, o-o] ++(0,-4) node [ground]{}
	;
\end{circuitikz}
\end{minted}
















\section{Diodes}


\subsection{Alone}
\begin{center}
\begin{circuitikz}\draw
	(0,0) node[anchor=east] {A} to [short,i>^=$I$] (1.5,0)
	(0,0) to [Do, v<=$V$] (2.5,0) node [anchor=west]{K}
;\end{circuitikz}
\end{center}

\begin{minted}{latex}
		\begin{circuitikz}\draw
			(0,0) node[anchor=east] {A} to [short,i>^=$I$] (1.5,0)
			(0,0) to [Do, v<=$V$] (2.5,0) node [anchor=west]{K}
		;\end{circuitikz}
\end{minted}



\subsection{Pulsed LED}
\begin{center}
\begin{circuitikz}\draw
	(0,0) to [square voltage source, l=$E1$] (0,2) to [R, l=$R$] (2,2) to [led, l_=$D$](2,0) --(0,0)
	;
\end{circuitikz}
\end{center}

\begin{minted}{latex}
		\begin{circuitikz}\draw
			(0,0) to [square voltage source, l=$E1$] (0,2) to [R, l=$R$] (2,2) to [led, l_=$D$](2,0) --(0,0)
			;
		\end{circuitikz}
\end{minted}


\subsection{LED}
\begin{center}
\begin{circuitikz}\draw
	(0,3) to [american voltage source, l=$V_{dc}$] (0,0)
	(0,3) to [european resistor, l^=$R$] (3,3)
	to [leDo] (3,0) -- (0,0)
;\end{circuitikz}
\end{center}

\begin{minted}{latex}
		\begin{circuitikz}\draw
			(0,3) to [american voltage source, l=$V_{dc}$] (0,0)
			(0,3) to [european resistor, l^=$R$] (3,3)
			to [leDo] (3,0) -- (0,0)
		;\end{circuitikz}
\end{minted}


\subsection{Load}
\begin{center}
\begin{circuitikz}\draw
	(0,0) to [sV, l=$V_{ac}$] (0,3)
	to [Do] (3,3)
	to [european resistor,l_=$Z_{Charge}$] (3,0) to (0,0)
	(3.5,3) to [open, v^<=$V_{charge}$] (3.5,0)
;\end{circuitikz}
\end{center}

\begin{minted}{latex}
		\begin{circuitikz}\draw
			(0,0) to [sV, l=$V_{ac}$] (0,3)
			to [Do] (3,3)
			to [european resistor,l_=$Z_{Charge}$] (3,0) to (0,0)
			(3.5,3) to [open, v^<=$V_{charge}$] (3.5,0)
		;\end{circuitikz}
\end{minted}


\subsection{Load and C in parallel}
\begin{center}
\begin{circuitikz}\draw
	(0,0) to [sV, l=$V_{ac}$] (0,3)
	to [Do] (5,3)
	to [european resistor,l=$R_{Ch}$] (5,0) to (0,0)
	(4,3) to [eC,l_=$C$, *-*] (4,0)
	(6,3) to [open, v^<=$V_{charge}$] (6,0)
;\end{circuitikz}
\end{center}

\begin{minted}{latex}
		\begin{circuitikz}\draw
			(0,0) to [sV, l=$V_{ac}$] (0,3)
			to [Do] (5,3)
			to [european resistor,l=$R_{Ch}$] (5,0) to (0,0)
			(4,3) to [eC,l_=$C$, *-*] (4,0)
			(6,3) to [open, v^<=$V_{charge}$] (6,0)
		;\end{circuitikz}
\end{minted}



\subsection{Full-wave rectifier with C and load}
\begin{center}
\begin{circuitikz}\draw
	(-2,1) to [sV, l_=$V_{ac}$] (-2,3)
	(-2,1) to (-1,1) to (-1,1.75) to [short,-*](2,1.75)
	(-2,3) to (-1,3) to (-1,2.25) to [short,-*](0,2.25)
	(0,0) to [Do] (0,2) to [Do](0,4)
	(2,0) to [Do,*-] (2,2) to [Do, -*](2,4)
	(0,4) to [short](5,4)
	(0,0) to [short](2,0)
	(5,4) to [european resistor,l=$Z_{Ch}$] (5,0) to (2,0)
	(4,4) to [eC,l_=$C$, *-*] (4,0)
	(6,4) to [open, v^<=$V_{charge}$] (6,0)
;\end{circuitikz}
\end{center}

\begin{minted}{latex}
		\begin{circuitikz}\draw
			(-2,1) to [sV, l_=$V_{ac}$] (-2,3)
			(-2,1) to (-1,1) to (-1,1.75) to [short,-*](2,1.75)
			(-2,3) to (-1,3) to (-1,2.25) to [short,-*](0,2.25)
			(0,0) to [Do] (0,2) to [Do](0,4)
			(2,0) to [Do,*-] (2,2) to [Do, -*](2,4)
			(0,4) to [short](5,4)
			(0,0) to [short](2,0)
			(5,4) to [european resistor,l=$Z_{Ch}$] (5,0) to (2,0)
			(4,4) to [eC,l_=$C$, *-*] (4,0)
			(6,4) to [open, v^<=$V_{charge}$] (6,0)
		;\end{circuitikz}
\end{minted}


\subsection{Zener alone}
\begin{center}
\begin{circuitikz}\draw
	(0,0) node[anchor=east] {A} to [short,i>^=$I$] (1.5,0)
	(0,0) to [zDo, v<=$V$] (2.5,0) node [anchor=west]{K}
;\end{circuitikz}
\end{center}


\begin{minted}{latex}
		\begin{circuitikz}\draw
			(0,0) node[anchor=east] {A} to [short,i>^=$I$] (1.5,0)
			(0,0) to [zDo, v<=$V$] (2.5,0) node [anchor=west]{K}
		;\end{circuitikz}
\end{minted}


\subsection{Zener - DC source}
\begin{center}
\begin{circuitikz}\draw
	(0,0) to [battery, invert, l=$V_{2}$] (0,3)
	to [european resistor,l=$R$] (3,3)
	(3,0) to [zDo, i<=$I_z$] (3,3)
	(3,0) to (0,0)
	(3,0) to [short,*-o] (4,0)
	(3,3) to [short,*-o] (4,3)
	(4,3) to [open,v^<=$V_{out}$] (4,0)
;\end{circuitikz}
\end{center}

\begin{minted}{latex}
		\begin{circuitikz}\draw
			(0,0) to [battery, invert, l=$V_{2}$] (0,3)
			to [european resistor,l=$R$] (3,3)
			(3,0) to [zDo, i<=$I_z$] (3,3)
			(3,0) to (0,0)
			(3,0) to [short,*-o] (4,0)
			(3,3) to [short,*-o] (4,3)
			(4,3) to [open,v^<=$V_{out}\equiv -V(Fig\ref{fig:zenerconv})$] (4,0)
		;\end{circuitikz}
\end{minted}


\subsection{Zener - DC source and load}
\begin{center}
\begin{circuitikz}\draw
	(0,3) to [battery, v_<=$V_{2}$] (0,0)
	(0,3) to [european resistor,l=$R$] (3,3)
	(3,0) to [zDo, i<=$I_z$] (3,3)
	(3,0) to (0,0)
	(3,0) to [short,*-o] (4,0) to (5,0)
	(3,3) to [short,*-o] (4,3) to [short, i=$I_{out}$] (5,3)
	(5,3) to [european resistor,l=$R_{\mbox{ch}}$] (5,0)
	(3.5,3) to [open,v^<=$V_{out}$] (3.5,0)
;\end{circuitikz}
\end{center}

\begin{minted}{latex}
		\begin{circuitikz}\draw
			(0,3) to [battery, v_<=$V_{2}$] (0,0)
			(0,3) to [european resistor,l=$R$] (3,3)
			(3,0) to [zDo, i<=$I_z$] (3,3)
			(3,0) to (0,0)
			(3,0) to [short,*-o] (4,0) to (5,0)
			(3,3) to [short,*-o] (4,3) to [short, i=$I_{out}$] (5,3)
			(5,3) to [european resistor,l=$R_{\mbox{ch}}$] (5,0)
			(3.5,3) to [open,v^<=$V_{out}$] (3.5,0)
		;\end{circuitikz}
\end{minted}












\section{Logic}

\subsection{Gates}
\begin{center}
\begin{circuitikz} \draw
	(0,0) node [american nand port]{}
	(-0.7,-0.8) node  {NAND}

	(2,0) node [american nor port] {}
	(2-0.7,-0.8) node  {NOR}

	(4,0) node [american xnor port] {}
	(4-0.7,-0.8) node  {XNOR}

	(0,2) node [american and port] {}
	(-0.7,2-0.8) node  {AND}

	(2,2) node [american or port] {}
	(2-0.7,2-0.8) node  {OR}

	(4,2) node [american xor port] {}
	(4-0.7,2-0.8) node  {XOR}

	(6,1) node [american not port] {}
	(6.7-0.7,1-0.8) node  {NOT}
;\end{circuitikz}
\end{center}

\begin{minted}{latex}
		\begin{circuitikz} \draw
			(0,0) node [american nand port]{}
			(-0.7,-0.8) node  {NAND}

			(2,0) node [american nor port] {}
			(2-0.7,-0.8) node  {NOR}

			(4,0) node [american xnor port] {}
			(4-0.7,-0.8) node  {XNOR}

			(0,2) node [american and port] {}
			(-0.7,2-0.8) node  {AND}

			(2,2) node [american or port] {}
			(2-0.7,2-0.8) node  {OR}

			(4,2) node [american xor port] {}
			(4-0.7,2-0.8) node  {XOR}

			(6,1) node [american not port] {}
			(6.7-0.7,1-0.8) node  {NOT}
		;\end{circuitikz}
\end{minted}


\subsection{Circuit 1}
\begin{center}
\begin{circuitikz} \draw
	(0,0) node [american nor port] (nor) {}
	(nor.in 2) -- ++(-1,0) node [ocirc] {} node [anchor=east] {C}
	(nor.in 1) -|  (-1.5,1)
	(0,2) node [american and port] (and){}
	(and.in 2) -| (-1.5,1) to [short, -o] (-2.4,1) node [anchor=east] {B}
	(and.in 1)	-- ++(-1,0) node [ocirc] {} node [anchor=east] {A}
	(2,1) node [american or port] (or){}
	(and.out) -| (or.in 1)
	(nor.out) -| (or.in 2)
	(3.5,1) node [american not port] (not){}
	(or.out) -- (not.in)
	(not.out) -- ++(1,0) node [ocirc] {} node [anchor=west] {X}
;\end{circuitikz}
\end{center}

\begin{minted}{latex}
		\begin{circuitikz} \draw
			(0,0) node [american nor port] (nor) {}
			(nor.in 2) -- ++(-1,0) node [ocirc] {} node [anchor=east] {C}
			(nor.in 1) -|  (-1.5,1)
			(0,2) node [american and port] (and){}
			(and.in 2) -| (-1.5,1) to [short, -o] (-2.4,1) node [anchor=east] {B}
			(and.in 1)	-- ++(-1,0) node [ocirc] {} node [anchor=east] {A}
			(2,1) node [american or port] (or){}
			(and.out) -| (or.in 1)
			(nor.out) -| (or.in 2)
			(3.5,1) node [american not port] (not){}
			(or.out) -- (not.in)
			(not.out) -- ++(1,0) node [ocirc] {} node [anchor=west] {X}
		;\end{circuitikz}
\end{minted}


\subsection{Voter}
\begin{center}
\begin{circuitikz} \draw
	(0,0) node [american and port] (and1) {}
	(and1.in 2) -- ++(-0.5,0) |- node [circ] {} ++(-0.5,2.56) node [ocirc] (B) {} node [anchor=east] {B}
	(0,2) node [american or port] (or){}
	(or.in 1) -- ++(-0.5,0) |- (B)
	(or.in 2) |- node [circ] {} ++(-1,-0.4) node [ocirc] (C) {} node [anchor=east] {C}
	(and1.in 1) |- (C)
	(2,3) node [american and port] (and2) {}
	(or.out) -| (and2.in 2)
	(and2.in 1) -- ++(-3,0)  node [ocirc] (A) {} node [anchor=east] {A}
	(or.out) -| (and2.in 2)
	(3.5,1) node [american or port] (and3){}
	(and2.out) -| (and3.in 1)
	(and1.out) -| (and3.in 2)
	(and3.out) -- ++(1,0) node [ocirc] {} node [anchor=west] {Y}
;\end{circuitikz}
\end{center}

\begin{minted}{latex}
		\begin{circuitikz} \draw
			(0,0) node [american and port] (and1) {}
			(and1.in 2) -- ++(-0.5,0) |- node [circ] {} ++(-0.5,2.56) node [ocirc] (B) {} node [anchor=east] {B}
			(0,2) node [american or port] (or){}
			(or.in 1) -- ++(-0.5,0) |- (B)
			(or.in 2) |- node [circ] {} ++(-1,-0.4) node [ocirc] (C) {} node [anchor=east] {C}
			(and1.in 1) |- (C)
			(2,3) node [american and port] (and2) {}
			(or.out) -| (and2.in 2)
			(and2.in 1) -- ++(-3,0)  node [ocirc] (A) {} node [anchor=east] {A}
			(or.out) -| (and2.in 2)
			(3.5,1) node [american or port] (and3){}
			(and2.out) -| (and3.in 1)
			(and1.out) -| (and3.in 2)
			(and3.out) -- ++(1,0) node [ocirc] {} node [anchor=west] {Y}
		;\end{circuitikz}
\end{minted}



\subsection{Circuit 2}

\begin{center}
\begin{circuitikz} \draw
	(0,0.72) node [american and port] (and1) {}
	(-3,1) node [american not port, scale=0.8] (not){}
	(and1.in 1) -|   (not.out)
	(not.in) |-  ++(-0.5,0) node [ocirc] (A) {} node [anchor=east] {A}
	(and1.in 2) |- ++(-2.66,0) node [ocirc] (B) {} node [anchor=east] {B}
	(and1.out) -- ++(1,0) node [ocirc] (Y) {} node [anchor=west] {Y}
;\end{circuitikz}
\end{center}

\begin{minted}{latex}
		\begin{circuitikz} \draw
			(0,0.72) node [american and port] (and1) {}
			(-3,1) node [american not port, scale=0.8] (not){}
			(and1.in 1) -|   (not.out)
			(not.in) |-  ++(-0.5,0) node [ocirc] (A) {} node [anchor=east] {A}
			(and1.in 2) |- ++(-2.66,0) node [ocirc] (B) {} node [anchor=east] {B}
			(and1.out) -- ++(1,0) node [ocirc] (Y) {} node [anchor=west] {Y}
		;\end{circuitikz}
\end{minted}


\subsection{Bistable}

\begin{center}
\begin{circuitikz} \draw
	(0,0) node [american not port] (not3) {}
	(2,0) node [american not port] (not4) {}
	(not3.out) -- (not4.in)
	(not4.out) -- ++(0.5,0) |- ++(-4,-1) |- (not3.in)
	(not3.out) |-  ++(0.25,0) node [circ] () {} node [anchor=south] {$\overline{Q}$}
	(not4.out) |-  ++(0.25,0) node [circ] () {} node [anchor=south] {$Q$}

	(8,1) node [american not port] (not1) {}
	(8,-1) node [american not port] (not2) {}
	(not1.out)  ++(0.5,-0.5)  coordinate (a-a) %coords of the crossing wire
	(not2.in)  ++(-1,0.5)  coordinate (a-b)

	(not1.in)++(-1.27,-0.5) node (in) {} % end of the wire with kinky bump
	(not2.out)-| ++(0.5,0.5) to  [kinky cross=(a-a)--(a-b), kinky crosses=left] (in)
	(not1.in)-| ++(-1.14,-0.55)
	(not2.out) -- ++(1,0) node [circ] () {} node [anchor=south] {$\overline{Q}$}

	(not1.out) -| ++(0.5,-0.5) -- ++(-3.043,-1) |- (not2.in)
	(not1.out) -- ++(1,0) node [circ] () {} node [anchor=south] {$Q$}
;\end{circuitikz}
\end{center}

\begin{minted}{latex}
\begin{circuitikz} \draw
	(0,0) node [american not port] (not3) {}
	(2,0) node [american not port] (not4) {}
	(not3.out) -- (not4.in)
	(not4.out) -- ++(0.5,0) |- ++(-4,-1) |- (not3.in)
	(not3.out) |-  ++(0.25,0) node [circ] () {} node [anchor=south] {$\overline{Q}$}
	(not4.out) |-  ++(0.25,0) node [circ] () {} node [anchor=south] {$Q$}

	(8,1) node [american not port] (not1) {}
	(8,-1) node [american not port] (not2) {}
	(not1.out)  ++(0.5,-0.5)  coordinate (a-a) %coords of the crossing wire
	(not2.in)  ++(-1,0.5)  coordinate (a-b)

	(not1.in)++(-1.27,-0.5) node (in) {} % end of the wire with kinky bump
	(not2.out)-| ++(0.5,0.5) to  [kinky cross=(a-a)--(a-b), kinky crosses=left] (in)
	(not1.in)-| ++(-1.14,-0.55)
	(not2.out) -- ++(1,0) node [circ] () {} node [anchor=south] {$\overline{Q}$}

	(not1.out) -| ++(0.5,-0.5) -- ++(-3.043,-1) |- (not2.in)
	(not1.out) -- ++(1,0) node [circ] () {} node [anchor=south] {$Q$}
;\end{circuitikz}
\end{minted}



\subsection{Enable}

\begin{center}
\begin{circuitikz} \draw
	(0,0) node [american not port] (not1) {}
	(not1)+(0,.25) |- ++(-1.7,1) node [ocirc] () {} node [anchor=south] {$B$}
	(not1.in) -- ++(-1,0) node [ocirc] () {} node [anchor=south] {$A$}
	(not1.out) -- ++(1,0) node [ocirc] () {} node [anchor=south] {$Q$}
;\end{circuitikz}
\end{center}

\begin{minted}{latex}
\begin{circuitikz} \draw
	(0,0) node [american not port] (not1) {}
	(not1)+(0,.25) |- ++(-1.7,1) node [ocirc] () {} node [anchor=south] {$B$}
	(not1.in) -- ++(-1,0) node [ocirc] () {} node [anchor=south] {$A$}
	(not1.out) -- ++(1,0) node [ocirc] () {} node [anchor=south] {$Q$}
;\end{circuitikz}
\end{minted}



\subsection{Bistable with enable}

\begin{center}
\begin{circuitikz} \draw
	(0,0) node [american not port] (not3) {}
	(not3)+(0,.25) |- ++(-0.7,1) node [ocirc] () {} node [anchor=south] {$B1$}
	(2,0) node [american not port] (not4) {}
	(not4)+(0,.25) |- ++(-0.7,1) node [ocirc] () {} node [anchor=south] {$B2$}
	(not3.out) -- (not4.in)
	(not4.out) -- ++(0.5,0) |- ++(-4,-1) |- (not3.in)
	(not3.out) |-  ++(0.25,0) node [circ] () {} node [anchor=south] {$\overline{Q}$}
	(not4.out) |-  ++(0.25,0) node [circ] () {} node [anchor=south] {$Q$}

	(8,1) node [american not port] (not1) {}
	(not1)+(0,0.25) |- ++(-0.7,1) node [ocirc] () {} node [anchor=south] {$B2$}
	(8,-1) node [american not port] (not2) {}
	(not2)+(0,-0.25) |- ++(-0.7,-1) node [ocirc] () {} node [anchor=north] {$B1$}
	(not1.out)  ++(0.5,-0.5)  coordinate (a-a) %coords of the crossing wire
	(not2.in)  ++(-1,0.5)  coordinate (a-b)

	(not1.in)++(-1.27,-0.5) node (in) {} % end of the wire with kinky bump
	(not2.out)-| ++(0.5,0.5) to  [kinky cross=(a-a)--(a-b), kinky crosses=left] (in)
	(not1.in)-| ++(-1.14,-0.55)
	(not2.out) -- ++(1,0) node [ocirc] () {} node [anchor=south] {$\overline{Q}$}

	(not1.out) -| ++(0.5,-0.5) -- ++(-3.043,-1) |- (not2.in)
	(not1.out) -- ++(1,0) node [ocirc] () {} node [anchor=south] {$Q$}
;\end{circuitikz}
\end{center}

\begin{minted}{latex}
\begin{circuitikz} \draw
	(0,0) node [american not port] (not3) {}
	(not3)+(0,.25) |- ++(-0.7,1) node [ocirc] () {} node [anchor=south] {$B1$}
	(2,0) node [american not port] (not4) {}
	(not4)+(0,.25) |- ++(-0.7,1) node [ocirc] () {} node [anchor=south] {$B2$}
	(not3.out) -- (not4.in)
	(not4.out) -- ++(0.5,0) |- ++(-4,-1) |- (not3.in)
	(not3.out) |-  ++(0.25,0) node [circ] () {} node [anchor=south] {$\overline{Q}$}
	(not4.out) |-  ++(0.25,0) node [circ] () {} node [anchor=south] {$Q$}

	(8,1) node [american not port] (not1) {}
	(not1)+(0,0.25) |- ++(-0.7,1) node [ocirc] () {} node [anchor=south] {$B2$}
	(8,-1) node [american not port] (not2) {}
	(not2)+(0,-0.25) |- ++(-0.7,-1) node [ocirc] () {} node [anchor=north] {$B1$}
	(not1.out)  ++(0.5,-0.5)  coordinate (a-a) %coords of the crossing wire
	(not2.in)  ++(-1,0.5)  coordinate (a-b)

	(not1.in)++(-1.27,-0.5) node (in) {} % end of the wire with kinky bump
	(not2.out)-| ++(0.5,0.5) to  [kinky cross=(a-a)--(a-b), kinky crosses=left] (in)
	(not1.in)-| ++(-1.14,-0.55)
	(not2.out) -- ++(1,0) node [ocirc] () {} node [anchor=south] {$\overline{Q}$}

	(not1.out) -| ++(0.5,-0.5) -- ++(-3.043,-1) |- (not2.in)
	(not1.out) -- ++(1,0) node [ocirc] () {} node [anchor=south] {$Q$}
;\end{circuitikz}
\end{minted}



\subsection{NOR}
\begin{center}
\begin{circuitikz} \draw
	(0,0) node [american nor port] (nor) {}
	(nor.in 1) -- ++(-1,0) node [ocirc] () {} node [anchor=east] {$A$}
	(nor.in 2) -- ++(-1,0) node [ocirc] () {} node [anchor=east] {$B$}
	(nor.out) -- ++(1,0) node [ocirc] () {} node [anchor=west] {$Q$}
;\end{circuitikz}
\end{center}

\begin{minted}{latex}
\begin{circuitikz} \draw
	(0,0) node [american nor port] (nor) {}
	(nor.in 1) -- ++(-1,0) node [ocirc] () {} node [anchor=east] {$A$}
	(nor.in 2) -- ++(-1,0) node [ocirc] () {} node [anchor=east] {$B$}
	(nor.out) -- ++(1,0) node [ocirc] () {} node [anchor=west] {$Q$}
;\end{circuitikz}
\end{minted}





\subsection{SR using NOR}
\begin{center}
\begin{circuitikz} \draw
	(0,1) node [american nor port] (nor1) {}
	(0,-1.5) node [american nor port] (nor2) {}
	(nor1.out)  ++(0.5,-0.5)  coordinate (a-a) %coords of the crossing wire
	(nor2.in 2)  ++(-1.5,0.5)  coordinate (a-b)

	(nor1.in 2)++(-1.135,-0.225) node (in) {} % end of the wire with kinky bump
	(nor2.out)-| ++(0.5,0.5) to  [kinky cross=(a-a)--(a-b), kinky crosses=left] (in)
	(nor1.in 2)-| ++(-1,-0.3)
	(nor1.out) -- ++(1.5,0) node [ocirc] () {} node [anchor=west] {$\overline{Q}$}

	(nor1.out) -| ++(0.5,-0.5) -- ++(-3.043,-1.5) |- (nor2.in 1)
	(nor2.out) -- ++(1.5,0) node [ocirc] () {} node [anchor=west] {$Q$}
	(nor1.out) |-  ++(0.25,0)

	(nor1.in 1) -- ++(-2,0) node [ocirc] () {} node [anchor=east] {$S$}
	(nor2.in 2) -- ++(-2,0) node [ocirc] () {} node [anchor=east] {$R$}
;\draw [dashed](-2.75,-2.25) rectangle (1,1.75);
\end{circuitikz}
\end{center}

\begin{minted}{latex}
\begin{circuitikz} \draw
	(0,1) node [american nor port] (nor1) {}
	(0,-1.5) node [american nor port] (nor2) {}
	(nor1.out)  ++(0.5,-0.5)  coordinate (a-a) %coords of the crossing wire
	(nor2.in 2)  ++(-1.5,0.5)  coordinate (a-b)

	(nor1.in 2)++(-1.135,-0.225) node (in) {} % end of the wire with kinky bump
	(nor2.out)-| ++(0.5,0.5) to  [kinky cross=(a-a)--(a-b), kinky crosses=left] (in)
	(nor1.in 2)-| ++(-1,-0.3)
	(nor1.out) -- ++(1.5,0) node [ocirc] () {} node [anchor=west] {$\overline{Q}$}

	(nor1.out) -| ++(0.5,-0.5) -- ++(-3.043,-1.5) |- (nor2.in 1)
	(nor2.out) -- ++(1.5,0) node [ocirc] () {} node [anchor=west] {$Q$}
	(nor1.out) |-  ++(0.25,0)

	(nor1.in 1) -- ++(-2,0) node [ocirc] () {} node [anchor=east] {$S$}
	(nor2.in 2) -- ++(-2,0) node [ocirc] () {} node [anchor=east] {$R$}
;\draw [dashed](-2.75,-2.25) rectangle (1,1.75);
\end{circuitikz}
\end{minted}



\subsection{SR using NAND}
\begin{center}
\begin{circuitikz} \draw
	(0,1) node [american nand port] (nor1) {}
	(0,-1.5) node [american nand port] (nor2) {}
	(nor1.out)  ++(0.5,-0.5)  coordinate (a-a) %coords of the crossing wire
	(nor2.in 2)  ++(-1.5,0.5)  coordinate (a-b)

	(nor1.in 2)++(-1.135,-0.225) node (in) {} % end of the wire with kinky bump
	(nor2.out)-| ++(0.5,0.5) to  [kinky cross=(a-a)--(a-b), kinky crosses=left] (in)
	(nor1.in 2)-| ++(-1,-0.3)
	(nor1.out) -- ++(1.5,0) node [ocirc] () {} node [anchor=west] {$Q$}

	(nor1.out) -| ++(0.5,-0.5) -- ++(-3.043,-1.5) |- (nor2.in 1)
	(nor2.out) -- ++(1.5,0) node [ocirc] () {} node [anchor=west] {$\overline{Q}$}
	(nor1.out) |-  ++(0.25,0)

	(nor1.in 1) -- ++(-2,0) node [ocirc] () {} node [anchor=east] {$\overline{S}$}
	(nor2.in 2) -- ++(-2,0) node [ocirc] () {} node [anchor=east] {$\overline{R}$}
;
\draw [dashed](-2.75,-2.25) rectangle (1,1.75);
\end{circuitikz}
\end{center}

\begin{minted}{latex}
\begin{circuitikz} \draw
	(0,1) node [american nand port] (nor1) {}
	(0,-1.5) node [american nand port] (nor2) {}
	(nor1.out)  ++(0.5,-0.5)  coordinate (a-a) %coords of the crossing wire
	(nor2.in 2)  ++(-1.5,0.5)  coordinate (a-b)

	(nor1.in 2)++(-1.135,-0.225) node (in) {} % end of the wire with kinky bump
	(nor2.out)-| ++(0.5,0.5) to  [kinky cross=(a-a)--(a-b), kinky crosses=left] (in)
	(nor1.in 2)-| ++(-1,-0.3)
	(nor1.out) -- ++(1.5,0) node [ocirc] () {} node [anchor=west] {$Q$}

	(nor1.out) -| ++(0.5,-0.5) -- ++(-3.043,-1.5) |- (nor2.in 1)
	(nor2.out) -- ++(1.5,0) node [ocirc] () {} node [anchor=west] {$\overline{Q}$}
	(nor1.out) |-  ++(0.25,0)

	(nor1.in 1) -- ++(-2,0) node [ocirc] () {} node [anchor=east] {$\overline{S}$}
	(nor2.in 2) -- ++(-2,0) node [ocirc] () {} node [anchor=east] {$\overline{R}$}
;
\draw [dashed](-2.75,-2.25) rectangle (1,1.75);
\end{circuitikz}
\end{minted}




\subsection{SR with NAND and enable}
\begin{center}
\begin{circuitikz} \draw
	(0,1.28) node [american nand port] (nand1) {}
	(0,-1.5-0.28) node [american nand port] (nand2) {}

	(nand1.in 1) -- ++(-1.5,0) node [ocirc] () {} node [anchor=east] {$S$}
	(nand2.in 2) -- ++(-1.5,0) node [ocirc] () {} node [anchor=east] {$R$}

	(nand1.in 2) |- ++(-1.5,-1.28) coordinate (dot) node [ocirc] () {} node [anchor=east] {$E$}
	(nand2.in 1) |- (dot)

	(3,1) node [american nand port] (nor1) {}
	(3,-1.5) node [american nand port] (nor2) {}
	(nor1.out)  ++(0.5,-0.5)  coordinate (a-a) %coords of the crossing wire
	(nor2.in 2)  ++(-1.5,0.5)  coordinate (a-b)

	(nor1.in 2)++(-1.135,-0.225) node (in) {} % end of the wire with kinky bump
	(nor2.out)-| ++(0.5,0.5) to  [kinky cross=(a-a)--(a-b), kinky crosses=left] (in)
	(nor1.in 2)-| ++(-1,-0.3)
	(nor1.out) -- ++(1.5,0) node [ocirc] () {} node [anchor=west] {$Q$}

	(nor1.out) -| ++(0.5,-0.5) -- ++(-3.043,-1.5) |- (nor2.in 1)
	(nor2.out) -- ++(1.5,0) node [ocirc] () {} node [anchor=west] {$\overline{Q}$}
	(nor1.out) |-  ++(0.25,0)
	(nor1.in 1) -| (nand1.out)
	(nor2.in 2) -| (nand2.out)
;\end{circuitikz}
\end{center}

\begin{minted}{latex}
\begin{circuitikz} \draw
	(0,1.28) node [american nand port] (nand1) {}
	(0,-1.5-0.28) node [american nand port] (nand2) {}

	(nand1.in 1) -- ++(-1.5,0) node [ocirc] () {} node [anchor=east] {$S$}
	(nand2.in 2) -- ++(-1.5,0) node [ocirc] () {} node [anchor=east] {$R$}

	(nand1.in 2) |- ++(-1.5,-1.28) coordinate (dot) node [ocirc] () {} node [anchor=east] {$E$}
	(nand2.in 1) |- (dot)

	(3,1) node [american nand port] (nor1) {}
	(3,-1.5) node [american nand port] (nor2) {}
	(nor1.out)  ++(0.5,-0.5)  coordinate (a-a) %coords of the crossing wire
	(nor2.in 2)  ++(-1.5,0.5)  coordinate (a-b)

	(nor1.in 2)++(-1.135,-0.225) node (in) {} % end of the wire with kinky bump
	(nor2.out)-| ++(0.5,0.5) to  [kinky cross=(a-a)--(a-b), kinky crosses=left] (in)
	(nor1.in 2)-| ++(-1,-0.3)
	(nor1.out) -- ++(1.5,0) node [ocirc] () {} node [anchor=west] {$Q$}

	(nor1.out) -| ++(0.5,-0.5) -- ++(-3.043,-1.5) |- (nor2.in 1)
	(nor2.out) -- ++(1.5,0) node [ocirc] () {} node [anchor=west] {$\overline{Q}$}
	(nor1.out) |-  ++(0.25,0)
	(nor1.in 1) -| (nand1.out)
	(nor2.in 2) -| (nand2.out)
;\end{circuitikz}
\end{minted}




\subsection{D latch}
\begin{center}
\begin{circuitikz} \draw
	(0,1.28) node [american nand port] (nand1) {}
	(0,-1.5-0.28) node [american nand port] (nand2) {}

	(nand1.in 1) -- ++(-2.5,0) node [ocirc] (D) {} node [anchor=east] {$D$}
	(-2.25,-2.07) node [american not port] (not) {}
	(D) -| (not.in)
	(not.out) --	(nand2.in 2)
	(D)++(0.94,0) coordinate (Dvert) %pour avoir seulement le segment vertical pour calculer l'intersection
	(nand1.in 2) |- ++(-1.25,-1.28) coordinate (dot)
	(dot) to [kinky cross=(Dvert)--(not.in), kinky crosses=left] ++(-1.25,0)node [ocirc] () {} node [anchor=east] {$E$}
	(nand2.in 1) |- (dot)

	(3,1) node [american nand port] (nor1) {}
	(3,-1.5) node [american nand port] (nor2) {}
	(nor1.out)  ++(0.5,-0.5)  coordinate (a-a) %coords of the crossing wire
	(nor2.in 2)  ++(-1.5,0.5)  coordinate (a-b)

	(nor1.in 2)++(-1.135,-0.225) node (in) {} % end of the wire with kinky bump
	(nor2.out)-| ++(0.5,0.5) to  [kinky cross=(a-a)--(a-b), kinky crosses=left] (in)
	(nor1.in 2)-| ++(-1,-0.3)
	(nor1.out) -- ++(1.5,0) node [ocirc] () {} node [anchor=west] {$Q$}

	(nor1.out) -| ++(0.5,-0.5) -- ++(-3.043,-1.5) |- (nor2.in 1)
	(nor2.out) -- ++(1.5,0) node [ocirc] () {} node [anchor=west] {$\overline{Q}$}
	(nor1.out) |-  ++(0.25,0)
	(nor1.in 1) -| (nand1.out)
	(nor2.in 2) -| (nand2.out)

;\draw [dashed](-3.5,-3.25) rectangle (4.25,2.75);
\end{circuitikz}
\end{center}

\begin{minted}{latex}
\begin{circuitikz} \draw
	(0,1.28) node [american nand port] (nand1) {}
	(0,-1.5-0.28) node [american nand port] (nand2) {}

	(nand1.in 1) -- ++(-2.5,0) node [ocirc] (D) {} node [anchor=east] {$D$}
	(-2.25,-2.07) node [american not port] (not) {}
	(D) -| (not.in)
	(not.out) --	(nand2.in 2)
	(D)++(0.94,0) coordinate (Dvert) %pour avoir seulement le segment vertical pour calculer l'intersection
	(nand1.in 2) |- ++(-1.25,-1.28) coordinate (dot)
	(dot) to [kinky cross=(Dvert)--(not.in), kinky crosses=left] ++(-1.25,0)node [ocirc] () {} node [anchor=east] {$E$}
	(nand2.in 1) |- (dot)

	(3,1) node [american nand port] (nor1) {}
	(3,-1.5) node [american nand port] (nor2) {}
	(nor1.out)  ++(0.5,-0.5)  coordinate (a-a) %coords of the crossing wire
	(nor2.in 2)  ++(-1.5,0.5)  coordinate (a-b)

	(nor1.in 2)++(-1.135,-0.225) node (in) {} % end of the wire with kinky bump
	(nor2.out)-| ++(0.5,0.5) to  [kinky cross=(a-a)--(a-b), kinky crosses=left] (in)
	(nor1.in 2)-| ++(-1,-0.3)
	(nor1.out) -- ++(1.5,0) node [ocirc] () {} node [anchor=west] {$Q$}

	(nor1.out) -| ++(0.5,-0.5) -- ++(-3.043,-1.5) |- (nor2.in 1)
	(nor2.out) -- ++(1.5,0) node [ocirc] () {} node [anchor=west] {$\overline{Q}$}
	(nor1.out) |-  ++(0.25,0)
	(nor1.in 1) -| (nand1.out)
	(nor2.in 2) -| (nand2.out)

;\draw [dashed](-3.5,-3.25) rectangle (4.25,2.75);
\end{circuitikz}
\end{minted}














\section{Graphs}
\subsection{Logarithmic axis}

\begin{center}
\begin{tikzpicture}
	\begin{loglogaxis}[
		xmin=1e-1, xmax=1e5,
		ymin=1e-1, ymax=1e5,
		yticklabels={,,},
		xticklabels={,,},
		grid=both,
		width=17cm,
		height=17cm,
		major grid style={black!50}
		]
	\end{loglogaxis}
\end{tikzpicture}
\end{center}

\begin{minted}{latex}
		\begin{tikzpicture}
			\begin{loglogaxis}[
				xmin=1e-1, xmax=1e5,
				ymin=1e-1, ymax=1e5,
				yticklabels={,,},
				xticklabels={,,},
				grid=both,
				width=17cm,
				height=17cm,
				major grid style={black!50}
				]
			\end{loglogaxis}
		\end{tikzpicture}
\end{minted}
\subsection{Semi-logarithmic axis}

\begin{center}
\begin{tikzpicture}
	\begin{axis}[
		xmode=log,
		xmin=1e-1, xmax=1e5,
		ymin=1, ymax=9,
		yticklabels={,,},
		xticklabels={,,},
		grid=both,
		width=17cm,
		height=9cm,
		major grid style={black!50}
		]
	\end{axis}
\end{tikzpicture}
\end{center}

\begin{minted}{latex}
		\begin{tikzpicture}
			\begin{axis}[
				xmode=log,
				xmin=1e-1, xmax=1e5,
				ymin=1, ymax=9,
				yticklabels={,,},
				xticklabels={,,},
				grid=both,
				width=17cm,
				height=9cm,
				major grid style={black!50}
				]
			\end{axis}
		\end{tikzpicture}
\end{minted}



\subsection{$I_z(V_z)$}
\begin{center}
\begin{tikzpicture}
	\begin{axis}[ %title ={4Hz Sine  Wave},
	% width=7cm,
	% height=5cm,
	axis lines=middle,
	% ymin=-10,
	ymax=4,
	xlabel ={$V_z$},
	xticklabels={},
	yticklabels={},
	% ytick={-10,-8,-6,-5,-4,-3,-2,-1,1,2,3,4,5,6,8,10}
	ylabel ={$I_z$},
    % grid=both,
    % grid style={line width=.1pt, draw=black!60},
    % major grid style={line width=.2pt,draw=black},
    % ultra thick,
    % minor tick num=5,
    % enlargelimits={abs=0.5},
    % axis line style={latex-latex},
    yticklabel style={font=\normalsize,fill=white},
    xlabel style={at={(ticklabel* cs:1)},anchor=north west},
    % ylabel style={at={(ticklabel* cs:1)},anchor=south west},
	]
	\addplot[%
	% red,
	domain=1:3,
	thick,
	samples=100
	]
	{0.9*(x-1)^2};
	% \addlegendentry{$V_{in}$}
	\addplot[%
	% red,
	domain=0:1,
	thick,
	samples=100
	]
	{0};
	\addplot[%
	red,
	domain=0:3,
	thick,
	samples=100
	]
	{-x+3};
	\addplot[%
	red,
	domain=0:2,
	thick,
	samples=100
	]
	{-1.5*x+3};
	\addplot[%
	red,
	domain=0:1,
	thick,
	samples=100
	]
	{-3*x+3};
	\addplot[%
	red,
	domain=0:0.5,
	thick,
	samples=100
	]
	{-6*x+3};
	\end{axis}
	% \draw[dashed] (4.55,0) -- (4.55,5);
	\draw[decorate, decoration={brace, amplitude=5pt}] ([yshift=-0.2cm]2.5,0)-- node[below=0.25cm, text width=2cm, align=center]
         {$V_2 < V_{BR}$}([yshift=-0.2cm]0,0); % Pour avoir une accolade avec la pointe vers le bas, d'abord donner la coordonnee de droite.
	\draw[decorate, decoration={brace, amplitude=5pt}] ([yshift=-0.2cm]6.85,0)-- node[below=0.25cm, text width=4cm, align=center]
         {$V_2 > V_{BR}$}([yshift=-0.2cm]2.5,0); % Pour avoir une accolade avec la pointe vers le bas, d'abord donner la coordonnee de droite.
    \draw [<-] (0,4.5) to [out=10,in=170] node[above]{$R_{ch} \searrow$} (6.85,4.5);
% Note that I had to replace the – by “to”.  Notice how the angles work:
% •
% When the curves goes “out” of (0,0), you put a needle with one extremity
% on the starting point and the other one facing right and you turn it coun-
% terclockwise until it is tangent to the curve.  The angle by which you have
% to turn the needle gives you the “out” angle.
% •
% When the curves goes “in” at (2,1.5), you put a needle with one extremity
% on the arrival point and the other one facing right and you turn it coun-
% terclockwise until it is tangent to the curve.  The angle by which you have
% to turn the needle gives you the “in” angle.
% https://cremeronline.com/LaTeX/minimaltikz.pdf
% A very minimal introduction to TikZ, by Jacques Cremer
\end{tikzpicture}
\end{center}

\begin{minted}{latex}
		\begin{tikzpicture}
			\begin{axis}[ %title ={4Hz Sine  Wave},
			% width=7cm,
			% height=5cm,
			axis lines=middle,
			% ymin=-10,
			ymax=4,
			xlabel ={$V_z$},
			xticklabels={},
			yticklabels={},
			% ytick={-10,-8,-6,-5,-4,-3,-2,-1,1,2,3,4,5,6,8,10}
			ylabel ={$I_z$},
		    % grid=both,
		    % grid style={line width=.1pt, draw=black!60},
		    % major grid style={line width=.2pt,draw=black},
		    % ultra thick,
		    % minor tick num=5,
		    % enlargelimits={abs=0.5},
		    % axis line style={latex-latex},
		    yticklabel style={font=\normalsize,fill=white},
		    xlabel style={at={(ticklabel* cs:1)},anchor=north west},
		    % ylabel style={at={(ticklabel* cs:1)},anchor=south west},
			]
			\addplot[%
			domain=1:3,
			thick,
			samples=100
			]
			{0.9*(x-1)^2};
			% \addlegendentry{$V_{in}$}
			\addplot[%
			domain=0:1,
			thick,
			samples=100
			]
			{0};
			\addplot[%
			red,
			domain=0:3,
			thick,
			samples=100
			]
			{-x+3};
			\addplot[%
			red,
			domain=0:2,
			thick,
			samples=100
			]
			{-1.5*x+3};
			\addplot[%
			red,
			domain=0:1,
			thick,
			samples=100
			]
			{-3*x+3};
			\addplot[%
			red,
			domain=0:0.5,
			thick,
			samples=100
			]
			{-6*x+3};
			\end{axis}
			% \draw[dashed] (4.55,0) -- (4.55,5);
			\draw[decorate, decoration={brace, amplitude=5pt}] ([yshift=-0.2cm]2.5,0)-- node[below=0.25cm, text width=2cm, align=center]
		         {$V_2 < V_{BR}$}([yshift=-0.2cm]0,0); % Pour avoir une accolade avec la pointe vers le bas, d'abord donner la coordonnee de droite.
			\draw[decorate, decoration={brace, amplitude=5pt}] ([yshift=-0.2cm]6.85,0)-- node[below=0.25cm, text width=4cm, align=center]
		         {$V_2 > V_{BR}$}([yshift=-0.2cm]2.5,0); % Pour avoir une accolade avec la pointe vers le bas, d'abord donner la coordonnee de droite.
		    \draw [<-] (0,4.5) to [out=10,in=170] node[above]{$R_{ch} \searrow$} (6.85,4.5);
		% Note that I had to replace the – by “to”.  Notice how the angles work:
		% •
		% When the curves goes “out” of (0,0), you put a needle with one extremity
		% on the starting point and the other one facing right and you turn it coun-
		% terclockwise until it is tangent to the curve.  The angle by which you have
		% to turn the needle gives you the “out” angle.
		% •
		% When the curves goes “in” at (2,1.5), you put a needle with one extremity
		% on the arrival point and the other one facing right and you turn it coun-
		% terclockwise until it is tangent to the curve.  The angle by which you have
		% to turn the needle gives you the “in” angle.
		% https://cremeronline.com/LaTeX/minimaltikz.pdf
		% A very minimal introduction to TikZ, by Jacques Cremer
		\end{tikzpicture}
\end{minted}



\subsection{$V_{out}(I_{out})$}
\begin{center}
\begin{tikzpicture}
	\begin{axis}[ %title ={4Hz Sine  Wave},
	% width=7cm,
	% height=5cm,
	axis lines=middle,
	% ymin=-10,
	ymax=1.5,
	xlabel ={$I_{out}$},
	xticklabels={},
	yticklabels={},
	% ytick={-10,-8,-6,-5,-4,-3,-2,-1,1,2,3,4,5,6,8,10}
	ylabel ={$V_{out}$},
    % grid=both,
    % grid style={line width=.1pt, draw=black!60},
    % major grid style={line width=.2pt,draw=black},
    % ultra thick,
    % minor tick num=5,
    % enlargelimits={abs=0.5},
    % axis line style={latex-latex},
    yticklabel style={font=\normalsize,fill=white},
    xlabel style={at={(ticklabel* cs:1)},anchor=north west},
    % ylabel style={at={(ticklabel* cs:1)},anchor=south west},
	]
	\addplot[%
	domain=0:2,
	thick,
	samples=100
	]
	{1};
	% \addlegendentry{$V_{in}$}
	\addplot[%
	domain=2:3,
	thick,
	samples=100
	]
	{-x+3};
	\end{axis}
	\draw[dashed] (4.55,0) -- (4.55,5);
	\draw[decorate, decoration={brace, amplitude=5pt}] ([yshift=-0.2cm]4.55,0)-- node[below=0.25cm, text width=4cm]
         {Grosse charge, $I_{out}$ est donc faible et la Zener est en avalanche. La charge est régulée.}([yshift=-0.2cm]0,0); % Pour avoir une accolade avec la pointe vers le bas, d'abord donner la coordonnee de droite.
	\draw[decorate, decoration={brace, amplitude=5pt}] ([yshift=-0.2cm]6.85,0)-- node[below=0.25cm, text width=2cm]
         {Diviseur résistif, la Zener est bloquante.}([yshift=-0.2cm]4.55,0); % Pour avoir une accolade avec la pointe vers le bas, d'abord donner la coordonnee de droite.
    \draw [->] (0,4.5) to [out=10,in=170] node[above]{$R_{ch} \searrow$} (6.85,4.5);
% Note that I had to replace the – by “to”.  Notice how the angles work:
% •
% When the curves goes “out” of (0,0), you put a needle with one extremity
% on the starting point and the other one facing right and you turn it coun-
% terclockwise until it is tangent to the curve.  The angle by which you have
% to turn the needle gives you the “out” angle.
% •
% When the curves goes “in” at (2,1.5), you put a needle with one extremity
% on the arrival point and the other one facing right and you turn it coun-
% terclockwise until it is tangent to the curve.  The angle by which you have
% to turn the needle gives you the “in” angle.
% https://cremeronline.com/LaTeX/minimaltikz.pdf
% A very minimal introduction to TikZ, by Jacques Cremer
\end{tikzpicture}
\end{center}

\begin{minted}{latex}
		\begin{tikzpicture}
			\begin{axis}[ %title ={4Hz Sine  Wave},
			% width=7cm,
			% height=5cm,
			axis lines=middle,
			% ymin=-10,
			ymax=1.5,
			xlabel ={$I_{out}$},
			xticklabels={},
			yticklabels={},
			% ytick={-10,-8,-6,-5,-4,-3,-2,-1,1,2,3,4,5,6,8,10}
			ylabel ={$V_{out}$},
		    % grid=both,
		    % grid style={line width=.1pt, draw=black!60},
		    % major grid style={line width=.2pt,draw=black},
		    % ultra thick,
		    % minor tick num=5,
		    % enlargelimits={abs=0.5},
		    % axis line style={latex-latex},
		    yticklabel style={font=\normalsize,fill=white},
		    xlabel style={at={(ticklabel* cs:1)},anchor=north west},
		    % ylabel style={at={(ticklabel* cs:1)},anchor=south west},
			]
			\addplot[%
			domain=0:2,
			thick,
			samples=100
			]
			{1};
			% \addlegendentry{$V_{in}$}
			\addplot[%
			domain=2:3,
			thick,
			samples=100
			]
			{-x+3};
			\end{axis}
			\draw[dashed] (4.55,0) -- (4.55,5);
			\draw[decorate, decoration={brace, amplitude=5pt}] ([yshift=-0.2cm]4.55,0)-- node[below=0.25cm, text width=4cm]
		         {Grosse charge, $I_{out}$ est donc faible et la Zener est en avalanche. La charge est régulée.}([yshift=-0.2cm]0,0); % Pour avoir une accolade avec la pointe vers le bas, d'abord donner la coordonnee de droite.
			\draw[decorate, decoration={brace, amplitude=5pt}] ([yshift=-0.2cm]6.85,0)-- node[below=0.25cm, text width=2cm]
		         {Diviseur résistif, la Zener est bloquante.}([yshift=-0.2cm]4.55,0); % Pour avoir une accolade avec la pointe vers le bas, d'abord donner la coordonnee de droite.
		    \draw [->] (0,4.5) to [out=10,in=170] node[above]{$R_{ch} \searrow$} (6.85,4.5);
		% Note that I had to replace the – by “to”.  Notice how the angles work:
		% •
		% When the curves goes “out” of (0,0), you put a needle with one extremity
		% on the starting point and the other one facing right and you turn it coun-
		% terclockwise until it is tangent to the curve.  The angle by which you have
		% to turn the needle gives you the “out” angle.
		% •
		% When the curves goes “in” at (2,1.5), you put a needle with one extremity
		% on the arrival point and the other one facing right and you turn it coun-
		% terclockwise until it is tangent to the curve.  The angle by which you have
		% to turn the needle gives you the “in” angle.
		% https://cremeronline.com/LaTeX/minimaltikz.pdf
		% A very minimal introduction to TikZ, by Jacques Cremer
		\end{tikzpicture}
\end{minted}


\subsection{Time graph 1}

\begin{center}
\begin{tikzpicture}[minimum size= 1.25cm,scale=0.9]
\usetikzlibrary{calc}
{
	\draw   [->]  (0,0) -- (0,1);
	\node [anchor=east] at (0,1) {A};
	\draw [->]( 0,0) -- (10.5,0);
	\node [anchor=west] at (10.5,0) {t};

	\foreach \x in {1,2,...,10} \draw (\x,-0.1) -- (\x,0.1);
	\foreach \x in {1,2,...,10} \draw (\x,-0.1-2) -- (\x,0.1-2);
	\foreach \x in {1,2,...,10} \draw (\x,-0.1-4) -- (\x,0.1-4);
	\node [anchor=north, inner sep=0pt, outer sep=0pt] at (1,0.25) {10ns};
	\node [anchor=north, inner sep=0pt, outer sep=0pt] at (2,0.25) {20ns};

	\draw [->] (0,-2) -- (0,1-2);
	\node [anchor=east] at (0,1-2) {B};
	\draw [->] (0,-2) -- (10.5,-2);
	\node [anchor=west] at (10.5,-2) {t};

	\draw [->] (0,-4) -- (0,1-4);
	\node [anchor=east] at (0,1-4) {Y};
	\draw [->] (0,-4) -- (10.5,-4);
	\node [anchor=west] at (10.5,-4) {t};

	\draw [line width=2pt] (0,0) -|(3,1) -| (10,1); %A
	\draw [line width=2pt] (0,0-2) -|(1,1-2) -| (5,0-2) -- (10,0-2); %B

	\draw [line width=2pt] (0,-4) -| (3,1-4) -|(6,0-4)--(10,0-4); %Y
}
\end{tikzpicture}
\end{center}

\begin{minted}{latex}
\usetikzlibrary{calc}
{
	\draw   [->]  (0,0) -- (0,1);
	\node [anchor=east] at (0,1) {A};
	\draw [->]( 0,0) -- (10.5,0);
	\node [anchor=west] at (10.5,0) {t};
	
	\foreach \x in {1,2,...,10} \draw (\x,-0.1) -- (\x,0.1);
	\foreach \x in {1,2,...,10} \draw (\x,-0.1-2) -- (\x,0.1-2);
	\foreach \x in {1,2,...,10} \draw (\x,-0.1-4) -- (\x,0.1-4);
	\node [anchor=north, inner sep=0pt, outer sep=0pt] at (1,0.25) {10ns};
	\node [anchor=north, inner sep=0pt, outer sep=0pt] at (2,0.25) {20ns};

	\draw [->] (0,-2) -- (0,1-2);
	\node [anchor=east] at (0,1-2) {B};
	\draw [->] (0,-2) -- (10.5,-2);
	\node [anchor=west] at (10.5,-2) {t};

	\draw [->] (0,-4) -- (0,1-4);
	\node [anchor=east] at (0,1-4) {Y};
	\draw [->] (0,-4) -- (10.5,-4);
	\node [anchor=west] at (10.5,-4) {t};

	\draw [line width=2pt] (0,0) -|(3,1) -| (10,1); %A
	\draw [line width=2pt] (0,0-2) -|(1,1-2) -| (5,0-2) -- (10,0-2); %B

	\draw [line width=2pt] (0,-4) -| (3,1-4) -|(6,0-4)--(10,0-4); %Y
}
\end{tikzpicture}
\end{minted}

















\section{Miscellaneous}

\subsection{74HC00}

\begin{center}
\begin{circuitikz}[scale=0.8] \draw
	(4,2.5) node [anchor=center] {$74HC00, 4\cdot NAND$}
	(2.8,1.5) node [american nand port,scale=0.8] (nand1) {}
	(1,0) node (in11) {}
	(2,0) node (in12) {}
	(3,0) node (out1) {}
	(in11) |- (nand1.in 1)
	(in12) |- ++(-0.6,0.75)|- (nand1.in 2)
	(out1) |- (nand1.out)

	(2.8+3,1.5) node [american nand port,scale=0.8] (nand2) {}
	(1+3,0) node (in21) {}
	(2+3,0) node (in22) {}
	(3+3,0) node (out2) {}
	(in21) |- (nand2.in 1)
	(in22) |- ++(-0.6,0.75)|- (nand2.in 2)
	(out2) |- (nand2.out)


	(1+2.8,5-1.5) node [american nand port,scale=0.8] (nand3) {}
	(1+1,5) node (in31) {}
	(2+1,5) node (in32) {}
	(3+1,5) node (out3) {}
	(in31) |- (nand3.in 2)
	(in32) |- ++(-0.6,-0.75)|- (nand3.in 1)
	(out3) |- (nand3.out)

	(1+2.8+3,5-1.5) node [american nand port,scale=0.8] (nand4) {}
	(2+3,5) node (in41) {}
	(3+3,5) node (in42) {}
	(4+3,5) node (out4) {}
	(in41) |- (nand4.in 2)
	(in42) |- ++(-0.6,-0.75)|- (nand4.in 1)
	(out4) |- (nand4.out)

	(7,0-0.25) node [anchor=north](gnd) {GND}
	(1,5+0.35) node [anchor=south](vcc) {VCC}

;\draw (0,0)rectangle (8,5);
\foreach \x in {1,2,...,7} \filldraw [fill=white] (\x-0.25,-0.15) rectangle (\x+0.25,0.35) (\x,0.1) node {\x};
\foreach \x in {1,2,...,7} \filldraw [fill=white] (\x-0.25,5-0.15) rectangle (\x+0.25,5+0.35);
\foreach \x in {8,9,...,14} \draw (15-\x,5+0.1) node {\x};
\draw (0,2) arc[start angle=-90, end angle=90, radius=0.5];
\end{circuitikz}
\end{center}

\begin{minted}{latex}
\begin{circuitikz}[scale=0.8] \draw
	(4,2.5) node [anchor=center] {$74HC00, 4\cdot NAND$}
	(2.8,1.5) node [american nand port,scale=0.8] (nand1) {}
	(1,0) node (in11) {}
	(2,0) node (in12) {}
	(3,0) node (out1) {}
	(in11) |- (nand1.in 1)
	(in12) |- ++(-0.6,0.75)|- (nand1.in 2)
	(out1) |- (nand1.out)

	(2.8+3,1.5) node [american nand port,scale=0.8] (nand2) {}
	(1+3,0) node (in21) {}
	(2+3,0) node (in22) {}
	(3+3,0) node (out2) {}
	(in21) |- (nand2.in 1)
	(in22) |- ++(-0.6,0.75)|- (nand2.in 2)
	(out2) |- (nand2.out)


	(1+2.8,5-1.5) node [american nand port,scale=0.8] (nand3) {}
	(1+1,5) node (in31) {}
	(2+1,5) node (in32) {}
	(3+1,5) node (out3) {}
	(in31) |- (nand3.in 2)
	(in32) |- ++(-0.6,-0.75)|- (nand3.in 1)
	(out3) |- (nand3.out)

	(1+2.8+3,5-1.5) node [american nand port,scale=0.8] (nand4) {}
	(2+3,5) node (in41) {}
	(3+3,5) node (in42) {}
	(4+3,5) node (out4) {}
	(in41) |- (nand4.in 2)
	(in42) |- ++(-0.6,-0.75)|- (nand4.in 1)
	(out4) |- (nand4.out)

	(7,0-0.25) node [anchor=north](gnd) {GND}
	(1,5+0.35) node [anchor=south](vcc) {VCC}

;\draw (0,0)rectangle (8,5);
\foreach \x in {1,2,...,7} \filldraw [fill=white] (\x-0.25,-0.15) rectangle (\x+0.25,0.35) (\x,0.1) node {\x};
\foreach \x in {1,2,...,7} \filldraw [fill=white] (\x-0.25,5-0.15) rectangle (\x+0.25,5+0.35);
\foreach \x in {8,9,...,14} \draw (15-\x,5+0.1) node {\x};
\draw (0,2) arc[start angle=-90, end angle=90, radius=0.5];
\end{circuitikz}
\end{minted}



\subsection{74HC32}
\begin{center}
\begin{circuitikz}[scale=0.8] \draw
	(4,2.5) node [anchor=center] {$74HC32, 4\cdot  OR$}
	(2.8,1.5) node [american or port,scale=0.8] (or1) {}
	(1,0) node (in11) {}
	(2,0) node (in12) {}
	(3,0) node (out1) {}
	(in11) |- (or1.in 1)
	(in12) |- ++(-0.6,0.75)|- (or1.in 2)
	(out1) |- (or1.out)

	(2.8+3,1.5) node [american or port,scale=0.8] (or2) {}
	(1+3,0) node (in21) {}
	(2+3,0) node (in22) {}
	(3+3,0) node (out2) {}
	(in21) |- (or2.in 1)
	(in22) |- ++(-0.6,0.75)|- (or2.in 2)
	(out2) |- (or2.out)


	(1+2.8,5-1.5) node [american or port,scale=0.8] (or3) {}
	(1+1,5) node (in31) {}
	(2+1,5) node (in32) {}
	(3+1,5) node (out3) {}
	(in31) |- (or3.in 2)
	(in32) |- ++(-0.6,-0.75)|- (or3.in 1)
	(out3) |- (or3.out)

	(1+2.8+3,5-1.5) node [american or port,scale=0.8] (or4) {}
	(2+3,5) node (in41) {}
	(3+3,5) node (in42) {}
	(4+3,5) node (out4) {}
	(in41) |- (or4.in 2)
	(in42) |- ++(-0.6,-0.75)|- (or4.in 1)
	(out4) |- (or4.out)

	(7,0-0.25) node [anchor=north](gnd) {GND}
	(1,5+0.35) node [anchor=south](vcc) {VCC}

;
\draw (0,0)rectangle (8,5);
\foreach \x in {1,2,...,7} \filldraw [fill=white] (\x-0.25,-0.15) rectangle (\x+0.25,0.35) (\x,0.1) node {\x};
\foreach \x in {1,2,...,7} \filldraw [fill=white] (\x-0.25,5-0.15) rectangle (\x+0.25,5+0.35);
\foreach \x in {8,9,...,14} \draw (15-\x,5+0.1) node {\x};
\draw (0,2) arc[start angle=-90, end angle=90, radius=0.5];
\end{circuitikz}
\end{center}

\begin{minted}{latex}
\begin{circuitikz}[scale=0.8] \draw
	(4,2.5) node [anchor=center] {$74HC32, 4\cdot  OR$}
	(2.8,1.5) node [american or port,scale=0.8] (or1) {}
	(1,0) node (in11) {}
	(2,0) node (in12) {}
	(3,0) node (out1) {}
	(in11) |- (or1.in 1)
	(in12) |- ++(-0.6,0.75)|- (or1.in 2)
	(out1) |- (or1.out)

	(2.8+3,1.5) node [american or port,scale=0.8] (or2) {}
	(1+3,0) node (in21) {}
	(2+3,0) node (in22) {}
	(3+3,0) node (out2) {}
	(in21) |- (or2.in 1)
	(in22) |- ++(-0.6,0.75)|- (or2.in 2)
	(out2) |- (or2.out)


	(1+2.8,5-1.5) node [american or port,scale=0.8] (or3) {}
	(1+1,5) node (in31) {}
	(2+1,5) node (in32) {}
	(3+1,5) node (out3) {}
	(in31) |- (or3.in 2)
	(in32) |- ++(-0.6,-0.75)|- (or3.in 1)
	(out3) |- (or3.out)

	(1+2.8+3,5-1.5) node [american or port,scale=0.8] (or4) {}
	(2+3,5) node (in41) {}
	(3+3,5) node (in42) {}
	(4+3,5) node (out4) {}
	(in41) |- (or4.in 2)
	(in42) |- ++(-0.6,-0.75)|- (or4.in 1)
	(out4) |- (or4.out)

	(7,0-0.25) node [anchor=north](gnd) {GND}
	(1,5+0.35) node [anchor=south](vcc) {VCC}

;
\draw (0,0)rectangle (8,5);
\foreach \x in {1,2,...,7} \filldraw [fill=white] (\x-0.25,-0.15) rectangle (\x+0.25,0.35) (\x,0.1) node {\x};
\foreach \x in {1,2,...,7} \filldraw [fill=white] (\x-0.25,5-0.15) rectangle (\x+0.25,5+0.35);
\foreach \x in {8,9,...,14} \draw (15-\x,5+0.1) node {\x};
\draw (0,2) arc[start angle=-90, end angle=90, radius=0.5];
\end{circuitikz}
\end{minted}




\subsection{74HC04}
\begin{center}
\begin{circuitikz}[scale=0.8] \draw
	(4,2.5) node [anchor=center] {$74HC04, 6\cdot NOT$}
	(1.5,1) node [american not port,scale=0.55] (not1) {}
	(1,0) node (in11) {}
	(2,0) node (out1) {}
	(in11) |- (not1.in)
	(out1) |- (not1.out)

	(1.5+2,1) node [american not port,scale=0.55] (not2) {}
	(1+2,0) node (in21) {}
	(2+2,0) node (out2) {}
	(in21) |- (not2.in)
	(out2) |- (not2.out)

	(1.5+4,1) node [american not port,scale=0.55] (not5) {}
	(1+4,0) node (in51) {}
	(2+4,0) node (out5) {}
	(in51) |- (not5.in)
	(out5) |- (not5.out)

	(1+1.5,5-1) node [american not port,scale=0.55] (not3) {}
	(1+1,5) node (in31) {}
	(2+1,5) node (out3) {}
	(in31) |- (not3.in)
	(out3) |- (not3.out)

	(1+1.5+2,5-1) node [american not port,scale=0.55] (not4) {}
	(2+2,5) node (in41) {}
	(3+2,5) node (out4) {}
	(in41) |- (not4.in)
	(out4) |- (not4.out)

	(1+1.5+4,5-1) node [american not port,scale=0.55] (not6) {}
	(2+4,5) node (in61) {}
	(3+4,5) node (out6) {}
	(in61) |- (not6.in)
	(out6) |- (not6.out)

	(7,0-0.25) node [anchor=north](gnd) {GND}
	(1,5+0.35) node [anchor=south](vcc) {VCC}
;\draw (0,0)rectangle (8,5);
\foreach \x in {1,2,...,7} \filldraw [fill=white] (\x-0.25,-0.15) rectangle (\x+0.25,0.35) (\x,0.1) node {\x};
\foreach \x in {1,2,...,7} \filldraw [fill=white] (\x-0.25,5-0.15) rectangle (\x+0.25,5+0.35);
\foreach \x in {8,9,...,14} \draw (15-\x,5+0.1) node {\x};
\draw (0,2) arc[start angle=-90, end angle=90, radius=0.5];
\end{circuitikz}
\end{center}

\begin{minted}{latex}
\begin{circuitikz}[scale=0.8] \draw
	(4,2.5) node [anchor=center] {$74HC04, 6\cdot NOT$}
	(1.5,1) node [american not port,scale=0.55] (not1) {}
	(1,0) node (in11) {}
	(2,0) node (out1) {}
	(in11) |- (not1.in)
	(out1) |- (not1.out)

	(1.5+2,1) node [american not port,scale=0.55] (not2) {}
	(1+2,0) node (in21) {}
	(2+2,0) node (out2) {}
	(in21) |- (not2.in)
	(out2) |- (not2.out)

	(1.5+4,1) node [american not port,scale=0.55] (not5) {}
	(1+4,0) node (in51) {}
	(2+4,0) node (out5) {}
	(in51) |- (not5.in)
	(out5) |- (not5.out)

	(1+1.5,5-1) node [american not port,scale=0.55] (not3) {}
	(1+1,5) node (in31) {}
	(2+1,5) node (out3) {}
	(in31) |- (not3.in)
	(out3) |- (not3.out)

	(1+1.5+2,5-1) node [american not port,scale=0.55] (not4) {}
	(2+2,5) node (in41) {}
	(3+2,5) node (out4) {}
	(in41) |- (not4.in)
	(out4) |- (not4.out)

	(1+1.5+4,5-1) node [american not port,scale=0.55] (not6) {}
	(2+4,5) node (in61) {}
	(3+4,5) node (out6) {}
	(in61) |- (not6.in)
	(out6) |- (not6.out)

	(7,0-0.25) node [anchor=north](gnd) {GND}
	(1,5+0.35) node [anchor=south](vcc) {VCC}
;\draw (0,0)rectangle (8,5);
\foreach \x in {1,2,...,7} \filldraw [fill=white] (\x-0.25,-0.15) rectangle (\x+0.25,0.35) (\x,0.1) node {\x};
\foreach \x in {1,2,...,7} \filldraw [fill=white] (\x-0.25,5-0.15) rectangle (\x+0.25,5+0.35);
\foreach \x in {8,9,...,14} \draw (15-\x,5+0.1) node {\x};
\draw (0,2) arc[start angle=-90, end angle=90, radius=0.5];
\end{circuitikz}
\end{minted}






\newpage{}

\tableofcontents

\end{document}
